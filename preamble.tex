% The preamble begins here.
\usepackage[body={6in,9in},
top=1in, left=1.5in, nohead]{geometry}
\newcommand{\be}{\begin{enumerate}}
\newcommand{\ee}{\end{enumerate}}
\newcommand{\bi}{\begin{itemize}}
\newcommand{\ei}{\end{itemize}}
% the amsfonts are needed for mathbb and mathfrak characters
% I use them all the time
\usepackage{amsfonts}    
\usepackage{latexsym}
\usepackage{amsmath} 
\usepackage{setspace} % for doublespacing
\doublespacing
\usepackage{graphicx}  % if you need graphics
\newcommand{\N}{\mathbb{N}}
\newcommand{\Z}{\mathbb{Z}}
\newcommand{\Q}{\mathbb{Q}}
\newcommand{\R}{\mathbb{R}}
\newcommand{\C}{\mathbb{C}}

%%%%% Theorems, Definitions, Examples, etc. 
\usepackage{amsthm}  %the amsthm package is needed for the following
\theoremstyle{plain} % default.  Heading is bold, text italic.
\newtheorem{thm}{Theorem}
\newtheorem{lem}{Lemma}
\newtheorem{prop}{Proposition}
\newtheorem*{cor}{Corollary}
\newtheorem*{RNT}{The Rank-Nullity Theorem}
\newtheorem*{BT}{Bezout's Theorem}

\theoremstyle{definition}  % Heading is bold, text is roman 
\newtheorem{defn}{Definition}
\newtheorem{conj}{Conjecture}
\newtheorem{example}{Example}
\newtheorem{step}{Step}

\theoremstyle{remark}  % Heading is italic, text is roman
\newtheorem*{rem}{Remark}
\newtheorem*{note}{Note}
\newtheorem{case}{Case}
\newtheorem{claim}{Claim}

